%%%%%%%%%%%%%%%%%%%%%%%%%%%%%%%%%%%%
%% BCCN Conference 2011           %%
%% Abstract (client-side)         %%
%%%%%%%%%%%%%%%%%%%%%%%%%%%%%%%%%%%%

% TODO: Check guidelines with regard to links/references.
% TODO: Confirm length requirements.
% TODO: \author{?}.

\documentclass[a4paper]{article}
\usepackage{natbib}

\begin{document}

\title{From the workbench: Tools for interacting with structured
  neurophysiological data in the G-Node storage system}

\author{TBD}
\date{August 8, 2011}

\maketitle

% Abstract

Structured, efficient and secure storage of experimental data and
associated meta-information constitutes one of the most pressing
challenges in modern neuroinformatics, and does so particularly in
electrophysiology. The German INCF node aims to provide a full-stack
solution for this specific domain, consisting of server-side
infrastructure that holds data in an object model tailored to
field-relevant demands and exposes these entities through a convenient
HTTP-based RESTful interface, as well as a collection of client-side
tools which facilitate data consumption and enable integration into
various popular analysis environments such as MATLAB, Python and
R. Existing workflows and code are leveraged whenever possible: By
transparently managing the transition from G-Node object format to
native data structures and vice versa, our client libraries allow
scientists to benefit from a powerful, semantic storage system without
having to drastically alter established processes.

The object model, derived from Python's NEO, offers a carefully
designed set of abstractions capturing the most prevalent components
of electrophysiological recordings and ranging from high-level
containers (e.g., blocks or segments) down to fine-grained elements
(e.g., spikes, waveforms or raw signal data). In combination with the
light-weight mark-up language odML, this model represents a
comprehensive approach to describing electrophysiological data and
attached meta-data. In order to maximize client compatibility, access
to these objects is primarily mediated by a concise, RESTful API which
transmits data via HTTP and in form of machine- as well as
human-readable JSON objects. Any programming and analysis suite
capable of performing simple HTTP requests therefore gains the ability
to consume, modify and create data maintained in the G-Node storage
system.

Given the variety of analysis environments in use, we acknowledge that
a sensible storage solution needs to integrate with existing
toolchains, and must do so seamlessly. To this end, G-Node puts
forward a set of client libraries for Python, JVM languages (e.g.,
Java), MATLAB and R which make stored recordings readily accessible
from each ecosystem and hide virtually all required middleware
operations (e.g., transfer, parsing, caching, and so on). Key aim is a
natural user experience, adhering to each language's idiosyncrasies
and transforming G-Node objects into idiomatic and computationally
efficient local representations. For instance, while the Python
interface implements a close mapping between stored and local objects,
our MATLAB toolbox exploits the runtime's heavily optimised array
handling when rendering objects. Bindings to R, on the other hand,
present requested data as suitably laid out data frames. By employing
standard patterns in each environment, researchers retain
straightforward access to relevant libraries supplied by their
respective analysis tools. Transformations of this kind are
bidirectional: client libraries facilitate tagging and storing of
existing recordings and support re-upload of modified objects. A
powerful side-effect of this unified data store is therefore enhanced
inter-operability between languages---recordings are easily
manipulated by disparate client applications. In addition, we are
currently developing browser-based visualisation tools aiding
scientists in exploratory data analysis.

Crucially, the extent to which advanced features of the object model
are used is user-determined; each library's design emphasises the
importance of gradual integration. Thus, by providing unobtrusive yet
fully-featured access, our client utilities bring the advantages of
the G-Node data management system closer to the electrophysiologist's
workbench.

\end{document}
