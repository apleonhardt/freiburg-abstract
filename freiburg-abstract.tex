%%%%%%%%%%%%%%%%%%%%%%%%%%%%%%%%%%%%
%% BCCN Conference 2011           %%
%% Abstract (client-side)         %%
%%%%%%%%%%%%%%%%%%%%%%%%%%%%%%%%%%%%

% TODO: Check guidelines with regard to links/references
% TODO: Confirm length requirements

\documentclass[a4paper]{article}
\usepackage{natbib}

\begin{document}

\title{Workbench Perspective: Tools for interacting with
  structured neurophysiological data in the G-Node Storage System}

\author{TBD} % TODO: Authors/order/affiliations/contact
\date{August 8, 2011}

\maketitle

% Abstract

Efficient, structured and secure storage of experimental data and
associated meta-information constitutes one of the most pressing
challenges in modern neuroinformatics, and especially
electrophysiology. The German INCF node aims to provide a full-stack
solution for this particular domain, consisting of server-side
infrastructure holding data in an object model tailored to
field-specific demands and exposing these entities through a
convenient HTTP-based RESTful interface, as well as a collection of
client-side tools that facilitate data consumption and allow
integration into various popular analysis environments such as MATLAB,
Python and R. Existing workflows and code are leveraged whenever
possible: By transparently managing the transition from G-Node object
format to native data structures and vice versa, our client libraries
enable scientists to benefit from a powerful, semantic storage system
without having to drastically alter established processes.

Our object model builds on NEO and offers a carefully
designed set of abstractions capturing the most prevalent components
of electrophysiological recordings, ranging from high-level elements
(e.g., blocks or segments) down to fine-grained units (e.g., spikes,
waveforms or raw signal data). In combination with the light-weight
mark-up language odML, this model represents a comprehensive approach
to describing electrophysiological data and meta-data. In order to
maximize client compatibility, access to such structured data is
primarily mediated by a concise RESTful API which transmits data via
HTTP and in form of machine- as well as human-readable JSON
objects. Any programming and analysis suite capable of performing
simple HTTP requests therefore gains the ability to consume, modify
and create data within the G-Node storage system.

Given the variety of analysis environments in use, we acknowledge that
a sensible storage solution needs to integrate with existing
toolchains. To this end, G-Node puts forward a set of client libraries
for Python, JVM languages (e.g., Java), MATLAB and R which make data
maintained by G-Node readily accessible from each ecosystem and
encapsulate virtually all intermediate steps (e.g., transfer, parsing,
caching, and so on). Key aim is a natural user experience, adhering to
each language's idiosyncracies and transforming G-Node objects into
idiomatic and computationally efficient local representations. For
instance, while the Python interface retains a close mapping between
stored and local objects, our MATLAB toolbox exploits the runtime's
heavily optimised array handling, and R bindings render data as
suitably configured data frames. By employing standard idioms in each
environment, researchers retain straightforward access to relevant
libraries supplied by their respective analysis tools. Transformations
of this kind are bidirectional: client libraries facilitate tagging
and storing of existing recordings and support re-upload of modified
objects. A powerful side-effect of a unified data store is enhanced
inter-operability between languages---recordings are easily
manipulated by multiple disparate client applications. Crucially, the
extent to which advanced features of the object model are used is
fully user-determined; each library's design emphasises the importance
of gradual integration.

In addition, we are currently developing browser-based visualisation
tools utilising aiding scientists in exploratory data analysis. Our
client utilities bring the advantages of the G-Node data management system
closer to the electrophsyiologist's workbench.

\end{document}
