%%%%%%%%%%%%%%%%%%%%%%%%%%%%%%%%%%%%
%% BCCN Conference 2011           %%
%% Abstract (client side)         %%
%%%%%%%%%%%%%%%%%%%%%%%%%%%%%%%%%%%%

\documentclass[a4paper]{article}

\begin{document}

\title{From a scientist's perspective: Tools for interacting with
  structured neurophysiological data in the G-Node Storage System}

\author{TBD} % TODO: Authors/order/affiliations/contact
\date{August 8, 2011}

\maketitle

% Abstract

Efficient, structured and secure storage of experimental data and
associated meta-information constitutes one of the most pressing
challenges in modern neuroinformatics, and especially in
electrophysiology. The German INCF node aims to provide a full-stack
solution to this particular problem, consisting of server-side
infrastructure holding data in an object model tailored to
domain-specific demands and exposing these objects through a
convenient HTTP-based REST interface, as well as a collection of
client-side tools that facilitate data consumption and allow
integration into various popular analysis environments such as MATLAB,
Python and R. Existing workflows and code are leveraged to the
greatest extent possible: By transparently managing the transition
between the G-Node object format and native data structures, our
libraries enable scientists to benefit from a powerful, semantic
storage solution without having to reinvent the wheel.



\end{document}
