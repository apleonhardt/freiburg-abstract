%%%%%%%%%%%%%%%%%%%%%%%%%%%%%%%%%%%%
%% BCCN Conference 2011           %%
%% Abstract (client-side)         %%
%%%%%%%%%%%%%%%%%%%%%%%%%%%%%%%%%%%%

% TODO: Check guidelines with regard to links/references.
% TODO: Confirm length requirements.
% TODO: \author{?}.

\documentclass[a4paper]{article}
\usepackage{natbib}

\begin{document}

\title{Advanced Tools and Techniques for Interacting with Neurophysiological Data}

\author{

  Aljoscha Leonhardt,
  Andrey Sobolev,
  Philipp L. Rautenberg,
  Christian Garbers,
  Christian Kellner,
  Andreas Herz,
  Thomas Wachtler

}
\date{August 13, 2011}

\maketitle

% Abstract

Structured, efficient, and secure storage of experimental data and
associated meta-information constitutes one of the most pressing
challenges in modern neuroscience, and does so particularly in
electrophysiology. The German INCF node aims to provide a full-stack
solution for this specific domain, consisting of two layers. First, we
offer server-side infrastructure that holds gathered data in an object
model tailored to field-relevant demands and exposes these entities
through a convenient HTTP-based RESTful interface. Second, we make a
collection of client-side tools available which facilitate data
consumption and enable native integration into various popular
analysis environments. Existing workflow and code are leveraged
wherever possible: By transparently managing the transition from
G-Node object format to native data structures and vice versa, our
client libraries allow scientists to benefit from a powerful, semantic
storage system without having to alter established processes
drastically.

Given the variety of analysis environments in use, we acknowledge that
a sensible storage solution needs to integrate with existing tool
chains, and must do so seamlessly. To this end, G-Node puts forward a
set of client libraries for Python, JVM languages (e.g., Java),
MATLAB, and R that hide virtually all intermediate operations (e.g.,
HTTP transfer, JSON parsing, and caching) and thereby render stored
recordings readily accessible. Key aim is natural user experience that
adheres to each language's idiosyncrasies by converting G-Node objects
into idiomatic and computationally efficient local representations.

Our MATLAB toolbox, for instance, exploits the runtime's heavily
optimised array and matrix handling when rendering objects. Specific
entities are transferred into the workspace by means of a single
command, and behave like standard structures. This allows painless
interfacing with existing tools for plotting and analysis while
retaining critical information about experimental logic. Moreover,
scientists with extensive experience in MATLAB become productive right
away. Bindings to R, on the other hand, present requested data as
suitably laid out data frames, again emphasising clean integration
into standard patterns of the respective language. Such
transformations are bidirectional: our client libraries facilitate
tagging and storing of existing recordings, and support effortless
re-upload of modified objects. Recordings and their structure are
therefore easily shared between analysis tools and laboratories. In
addition, we are currently developing browser-based visualisation
tools aiding scientists during exploratory data analysis.

Crucially, the extent to which advanced features of the object model
are used is determined by the individual scientist. Thus, by providing
unobtrusive yet fully-featured access, our client utilities bring
advantages of the G-Node data management system closer to the
neurophysiologist's workbench.

\end{document}
